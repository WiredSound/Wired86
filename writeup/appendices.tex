\begin{landscape} \section{Example Instruction Decodings}
    \begin{table}[h]
        \resizebox{0.87\paperheight}{!} {\begin{tabular} { |*{11}{c|} } % 11 columns
            \hline
            Assembly & \multicolumn{3}{c}{Opcode Byte} & \multicolumn{3}{|c}{MOD-REG-RM Byte} & \multicolumn{2}{|c}{Displacement Byte(s)} & \multicolumn{2}{|c|}{Immediate Byte(s)} \\
            \hline
            Intel Syntax                            & Opcode                & D-bit (direction)     & W-bit (word)          & MOD                   & REG           & R/M               & Low           & High      & Low       & High \\
            \hline
            \texttt{add cx, bx}                     & \underline{000000}01  & 0                     & 1                     & 11                    & 011           & 001               & & & & \\
                                                    & add                   & REG is source         & word data size        & register addressing   & BX source     & CX destination    & & & & \\
            \hline
            \texttt{add al, [bx+5]}                 & \underline{000000}10  & 1                     & 0                     & 01                    & 000           & 111               & 00000101      & & & \\
                                                    & add                   & REG is destination    & byte data size        & byte displacement     & AL dest       & BX + displacement & displace by 5 & & & \\
            \hline
            \texttt{mov word [bp+0x1234], 0x5678}   & \underline{110001}11  & 1                     & 1                     & 10                    & 000           & 110               & 00110100      & 00010010  & 01111000  & 01010110 \\
                                                    & mov Ew, Iw            & irrelevant            & word data size        & word displacement     & unused        & BP + displacement & 0x34          & 0x12      & 0x78      & 0x56 \\
            \hline
        \end{tabular}}
        \caption{Table showing example instruction decodings.}
    \end{table}
\end{landscape}

\section{Instruction Set}
    \label{sec:instructset}
    \subsection{Instruction Arguments}
    \begin{table}[h]
        \begin{tabular} { | c | m{0.85\textwidth} | }
            \hline
            Argument & Meaning \\
            \hline
            G & General-purpose register specified by the REG component of the MOD-REG-R/M byte. \\
            \hline
            E & Either a general-purpose register (specified by the R/M component when in register addressing mode) or a memory address (with optional displacement from index). \\
            \hline
        \end{tabular}
    \end{table}

    \subsection{Instructions}
        Note that this is not a complete list of Intel 8086 instructions but rather a list of just those that are supported by this emulator.

    \begin{table}[h]
        \begin{tabular} { | c | c | c | c | m{0.38\textwidth} | }
            \hline
            Opcode & Assembly & Arguments & MOD-REG-R/M & Description \\
            \hline
            \texttt{000000dw} & \texttt{add} & G, E & \checkmark & Add two values without carrying. \\
            \hline
            \texttt{0000010w} & \texttt{add} & AL/AX, I & & Add two values without carrying using AL (when w=0) or AX (when w=1) and an immediate value. \\
            \hline
            \texttt{00000110} & \texttt{push} & ES & & Push the value of the extra segment register onto the stack. \\
            \hline
            \texttt{00000111} & \texttt{pop} & ES & & Pop value off the stack and store it in extra segment register. \\
            \hline
            \texttt{000100dw} & \texttt{and} & G, E & \checkmark & Perform bitwise AND operation. \\
            \hline
            \texttt{0010010w} & \texttt{add} & AL/AX, I & & Perform bitwise AND operation with first argument being AL or AX an the second an immediate value. \\
            \hline
            % ADC instructions...
            \hline
            \texttt{0010110} & \texttt{push} & SS & & Push stack segment register. \\
            \hline
            \texttt{0010111} & \texttt{pop} & SS & & Pop stack segment register. \\
            \hline
            % AND instructions...
            % XOR instructions...
            \texttt{0100000} & \texttt{inc} & AX & & Add 1 to AX register value. \\
            \hline
            \texttt{0100001} & \texttt{inc} & CX & & Add 1 to CX register value. \\
            \hline
            \texttt{0100010} & \texttt{inc} & DX & & Add 1 to DX register value. \\
            \hline
            \texttt{0100011} & \texttt{inc} & BX & & Add 1 to BX register value. \\
            \hline
            \texttt{0100100} & \texttt{inc} & SP & & Add 1 to stack pointer. \\
            \hline
            \texttt{0100101} & \texttt{inc} & BP & & Add 1 to base pointer. \\
            \hline
            \texttt{0100110} & \texttt{inc} & SI & & Add 1 to source index. \\
            \hline
            \texttt{0100111} & \texttt{inc} & DI & & Add 1 to destination index. \\
        \end{tabular}
    \end{table}

\section{Full Interview} \label{sec:full-interview}
    \subsection{Are you satisfied with the resources you currently have available for teaching about the low-level workings of computing systems?}
        Do you know, that I have long thought how unsatisfied I am with my computing resources for teaching. Don't go quote me word-for-word. I'm only really aware of the little man computer simulation - it's quite basic but it quite good for GCSE-level students. However there doesn't appear to be anything suitable for A Level that provides greater detail for higher-level students. Gap in the market, for sure.
    
    \subsection{Have you considered implementing practical demonstrations into such lessons?}
        Yes, I think practical demonstrates are a real benefit when trying to get an idea across to a class - a great asset. I tried with the little man computer however it can be rather difficult for students to follow and understand.

    \subsection{If so, did you find that it enhanced the learning experience of your students?}
        Definitely, it is potential be a very dry, abstract topic - so yes, if you can represent it visual, I think that really helps people to understand it.
    
    \subsection{Have you before considered performing demonstrations using more simplistic, early computer systems (whether physical or emulated) to help illustrating your teaching points?}
        (Too lazy) I have considered it yet I'm often uncertain as to where I could get such systems - whether I would have time to setup in a classroom. Any I could find tend to be expensive and unreliable (old technology).
    
    \subsection{What features would you look for in an emulator to make it most applicable for usage in a teaching environment?}
        I would like to see the assembly code and how that correlates into opcode, operands - some representation of this in memory. How the different registers are being used - what they're storing. Anything that provides a bit of detail, a real time view of what is going on - how programming instruction is temporarily stored and then carried out. In a way, the simpler the better - students need to clearly see that connection between a simple program and how it runs.
    
    \subsection{What would, in your opinion, be the ideal interface for such a piece of software?}
        I'm imagining a GUI on my screen that I can use to get a bit of a demonstration of how of the operation of the system. Maybe I could hover my mouse over something and get an indication of what that particular part does. Making it as user friendly as possible would also be ideal. Colour coding could be useful to see what is active at each moment - draw one's eye to the right bit at the right time. Some sort of menu system at the top?